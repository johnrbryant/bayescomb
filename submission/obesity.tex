% Template for PLoS
% Version 3.5 March 2018
%
% % % % % % % % % % % % % % % % % % % % % %
%
% -- IMPORTANT NOTE
%
% This template contains comments intended
% to minimize problems and delays during our production
% process. Please follow the template instructions
% whenever possible.
%
% % % % % % % % % % % % % % % % % % % % % % %
%
% Once your paper is accepted for publication,
% PLEASE REMOVE ALL TRACKED CHANGES in this file
% and leave only the final text of your manuscript.
% PLOS recommends the use of latexdiff to track changes during review, as this will help to maintain a clean tex file.
% Visit https://www.ctan.org/pkg/latexdiff?lang=en for info or contact us at latex@plos.org.
%
%
% There are no restrictions on package use within the LaTeX files except that
% no packages listed in the template may be deleted.
%
% Please do not include colors or graphics in the text.
%
% The manuscript LaTeX source should be contained within a single file (do not use \input, \externaldocument, or similar commands).
%
% % % % % % % % % % % % % % % % % % % % % % %
%
% -- FIGURES AND TABLES
%
% Please include tables/figure captions directly after the paragraph where they are first cited in the text.
%
% DO NOT INCLUDE GRAPHICS IN YOUR MANUSCRIPT
% - Figures should be uploaded separately from your manuscript file.
% - Figures generated using LaTeX should be extracted and removed from the PDF before submission.
% - Figures containing multiple panels/subfigures must be combined into one image file before submission.
% For figure citations, please use "Fig" instead of "Figure".
% See http://journals.plos.org/plosone/s/figures for PLOS figure guidelines.
%
% Tables should be cell-based and may not contain:
% - spacing/line breaks within cells to alter layout or alignment
% - do not nest tabular environments (no tabular environments within tabular environments)
% - no graphics or colored text (cell background color/shading OK)
% See http://journals.plos.org/plosone/s/tables for table guidelines.
%
% For tables that exceed the width of the text column, use the adjustwidth environment as illustrated in the example table in text below.
%
% % % % % % % % % % % % % % % % % % % % % % % %
%
% -- EQUATIONS, MATH SYMBOLS, SUBSCRIPTS, AND SUPERSCRIPTS
%
% IMPORTANT
% Below are a few tips to help format your equations and other special characters according to our specifications. For more tips to help reduce the possibility of formatting errors during conversion, please see our LaTeX guidelines at http://journals.plos.org/plosone/s/latex
%
% For inline equations, please be sure to include all portions of an equation in the math environment.
%
% Do not include text that is not math in the math environment.
%
% Please add line breaks to long display equations when possible in order to fit size of the column.
%
% For inline equations, please do not include punctuation (commas, etc) within the math environment unless this is part of the equation.
%
% When adding superscript or subscripts outside of brackets/braces, please group using {}.
%
% Do not use \cal for caligraphic font.  Instead, use \mathcal{}
%
% % % % % % % % % % % % % % % % % % % % % % % %
%
% Please contact latex@plos.org with any questions.
%
% % % % % % % % % % % % % % % % % % % % % % % %

\documentclass[10pt,letterpaper]{article}
\usepackage[top=0.85in,left=2.75in,footskip=0.75in]{geometry}

% amsmath and amssymb packages, useful for mathematical formulas and symbols
\usepackage{amsmath,amssymb}

% Use adjustwidth environment to exceed column width (see example table in text)
\usepackage{changepage}

% Use Unicode characters when possible
\usepackage[utf8x]{inputenc}

% textcomp package and marvosym package for additional characters
\usepackage{textcomp,marvosym}

% cite package, to clean up citations in the main text. Do not remove.
% \usepackage{cite}

% Use nameref to cite supporting information files (see Supporting Information section for more info)
\usepackage{nameref,hyperref}

% line numbers
\usepackage[right]{lineno}

% ligatures disabled
\usepackage{microtype}
\DisableLigatures[f]{encoding = *, family = * }

% color can be used to apply background shading to table cells only
\usepackage[table]{xcolor}

% array package and thick rules for tables
\usepackage{array}

% create "+" rule type for thick vertical lines
\newcolumntype{+}{!{\vrule width 2pt}}

% create \thickcline for thick horizontal lines of variable length
\newlength\savedwidth
\newcommand\thickcline[1]{%
  \noalign{\global\savedwidth\arrayrulewidth\global\arrayrulewidth 2pt}%
  \cline{#1}%
  \noalign{\vskip\arrayrulewidth}%
  \noalign{\global\arrayrulewidth\savedwidth}%
}

% \thickhline command for thick horizontal lines that span the table
\newcommand\thickhline{\noalign{\global\savedwidth\arrayrulewidth\global\arrayrulewidth 2pt}%
\hline
\noalign{\global\arrayrulewidth\savedwidth}}


% Remove comment for double spacing
%\usepackage{setspace}
%\doublespacing

% Text layout
\raggedright
\setlength{\parindent}{0.5cm}
\textwidth 5.25in
\textheight 8.75in

% Bold the 'Figure #' in the caption and separate it from the title/caption with a period
% Captions will be left justified
\usepackage[aboveskip=1pt,labelfont=bf,labelsep=period,justification=raggedright,singlelinecheck=off]{caption}
\renewcommand{\figurename}{Fig}

% Use the PLoS provided BiBTeX style
% \bibliographystyle{plos2015}

% Remove brackets from numbering in List of References
\makeatletter
\renewcommand{\@biblabel}[1]{\quad#1.}
\makeatother



% Header and Footer with logo
\usepackage{lastpage,fancyhdr,graphicx}
\usepackage{epstopdf}
%\pagestyle{myheadings}
\pagestyle{fancy}
\fancyhf{}
%\setlength{\headheight}{27.023pt}
%\lhead{\includegraphics[width=2.0in]{PLOS-submission.eps}}
\rfoot{\thepage/\pageref{LastPage}}
\renewcommand{\headrulewidth}{0pt}
\renewcommand{\footrule}{\hrule height 2pt \vspace{2mm}}
\fancyheadoffset[L]{2.25in}
\fancyfootoffset[L]{2.25in}
\lfoot{\today}

%% Include all macros below

\newcommand{\lorem}{{\bf LOREM}}
\newcommand{\ipsum}{{\bf IPSUM}}


% Pandoc citation processing
\newlength{\csllabelwidth}
\setlength{\csllabelwidth}{3em}
\newlength{\cslhangindent}
\setlength{\cslhangindent}{1.5em}
% for Pandoc 2.8 to 2.10.1
\newenvironment{cslreferences}%
  {}%
  {\par}
% For Pandoc 2.11+
\newenvironment{CSLReferences}[2] % #1 hanging-ident, #2 entry spacing
 {% don't indent paragraphs
  \setlength{\parindent}{0pt}
  % turn on hanging indent if param 1 is 1
  \ifodd #1 \everypar{\setlength{\hangindent}{\cslhangindent}}\ignorespaces\fi
  % set entry spacing
  \ifnum #2 > 0
  \setlength{\parskip}{#2\baselineskip}
  \fi
 }%
 {}
\usepackage{calc} % for calculating minipage widths
\newcommand{\CSLBlock}[1]{#1\hfill\break}
\newcommand{\CSLLeftMargin}[1]{\parbox[t]{\csllabelwidth}{#1}}
\newcommand{\CSLRightInline}[1]{\parbox[t]{\linewidth - \csllabelwidth}{#1}\break}
\newcommand{\CSLIndent}[1]{\hspace{\cslhangindent}#1}

\usepackage{setspace}\doublespacing
\usepackage{subcaption}



\usepackage{forarray}
\usepackage{xstring}
\newcommand{\getIndex}[2]{
  \ForEach{,}{\IfEq{#1}{\thislevelitem}{\number\thislevelcount\ExitForEach}{}}{#2}
}

\setcounter{secnumdepth}{0}

\newcommand{\getAff}[1]{
  \getIndex{#1}{BDL,IPSR,FM,DP}
}

\providecommand{\tightlist}{%
  \setlength{\itemsep}{0pt}\setlength{\parskip}{0pt}}

\begin{document}
\vspace*{0.2in}

% Title must be 250 characters or less.
\begin{flushleft}
{\Large
\textbf\newline{A Bayesian approach to combining multiple information
sources: Estimating and forecasting childhood obesity in
Thailand} % Please use "sentence case" for title and headings (capitalize only the first word in a title (or heading), the first word in a subtitle (or subheading), and any proper nouns).
}
\newline
% Insert author names, affiliations and corresponding author email (do not include titles, positions, or degrees).
\\
John Bryant\textsuperscript{\getAff{BDL}, \getAff{IPSR}},
Jongjit Rittirong\textsuperscript{\getAff{IPSR}}\textsuperscript{*},
Wichai Aekplakorn\textsuperscript{\getAff{FM}},
Ladda Mo-suwan\textsuperscript{\getAff{DP}},
Pimolpan Nitnara\textsuperscript{\getAff{IPSR}}\\
\bigskip
\textbf{\getAff{BDL}}Bayesian Demography Limited, Christchurch, New
Zealand\\
\textbf{\getAff{IPSR}}Institute for Population and Social Research,
Mahidol University, Salaya, Nakhorn Pathom, Thailand\\
\textbf{\getAff{FM}}Faculty of Medicine, Ramathibodi Hospital, Mahidol
University, Bangkok, Thailand\\
\textbf{\getAff{DP}}Department of Paediatrics, Faculty of Medicine,
Prince of Songkla University, Hat Yai, Songkhla, Thailand\\
\bigskip
* Corresponding author: jongjit.rit@mahidol.edu\\
\end{flushleft}
% Please keep the abstract below 300 words
\section*{Abstract}
We estimate and forecast childhood obesity by age, sex, region, and
urban-rural residence in Thailand, using a Bayesian approach to
combining multiple source of information. Our main sources of
information are survey data and administrative data, but we also make
use of informative prior distributions based on international estimates
of obesity trends and on expectations about smoothness. Although the
final model is complex, the difficulty of building and understanding the
model is reduced by the fact that it is composed of many smaller
submodels. For instance, the submodel describing trends in prevalences
is specified separately from the submodels describing errors in the data
sources. None of our Thai data sources has more than 7 time points.
However, by combining multiple data sources, we are able to fit
relatively complicated time series models. Our results suggest that
obesity prevalence has recently starting rising quickly among Thai
teenagers throughout the country, but has been stable among children
under 5 years old.

% Please keep the Author Summary between 150 and 200 words
% Use first person. PLOS ONE authors please skip this step.
% Author Summary not valid for PLOS ONE submissions.

\linenumbers

% Use "Eq" instead of "Equation" for equation citations.
\newpage

\hypertarget{introduction}{%
\section{Introduction}\label{introduction}}

Disaggregated estimates and forecasts of social, economic, and health
outcomes can support more equitable and effective public policy.
Disaggregated estimates and forecasts can be used to identify groups
that are being poorly served, to assess the feasibility of policy
targets, to provide evidence on the effectiveness of interventions, and
to help with priority-setting {[}1,2{]}.

Disaggregated estimates and forecasts do, however, required
disaggregated data. Assembling datasets with the required level of
detail can be difficult. Household surveys may have the variables
needed, but sample sizes are often too small to support the desired
level of disaggregation. Population censuses have large samples, but are
carried out infrequently. Administrative data or big data, such as tax
records or cellphone data, have large samples and high frequency, but
often miss parts of the target population, and have substantial
measurement errors {[}3{]}. A particular problem for disaggregated
forecasting is assembling long time series of input data. The greater
the level of disaggregation, the more frequently geographical
boundaries, and definitions of variables and target populations, change
{[}4{]}. When the time series of input data are short, forecasting is
challenging {[}5, Section 13.7{]}.

One way to meet the demands for disaggregated data is to combine
information from many different sources {[}6,7{]}. Ideally, the sources
should have complementary strengths and weaknesses, so that gaps in any
one source can be filled by others. In many applications, it is
necessary to be opportunistic: to design the models around the data that
are available. It is, nevertheless, important to base the models on an
explicit statistical framework, to properly account for uncertainty
{[}8{]} and for transparency and replicability.

Bayesian statistical models are particularly well suited to combining
multiple sources of information. The key distinction between Bayesian
statistics and frequentist statistics is that Bayesians are willing to
use probability distributions to represent the state of knowledge about
any uncertain quantity {[}9,10{]}. Probabilities act as a common unit of
measurement, allowing Bayesians to combine many sources and types of
information within the same model. In the final model presented in this
paper, for instance, we use probability distributions to represent
sampling variability, the plausible range for measurement errors, and
the plausible range for annual variation in obesity rates.

In this paper, we use a Bayesian approach to combining information from
multiple sources to produce disaggregated estimates and forecasts of
childhood obesity in Thailand. The application is an important one.
Obesity has emerged as a major health issue throughout the world,
including in middle income countries such as Thailand {[}11,12{]}.
Disaggregated estimates provide information on the size of the problem,
and on risk factors: whether boys are more at risk than girls, for
instance, or whether urban children are more at risk than rural
children. Disaggregated forecasts help with planning and
priority-setting by health agencies by showing the scale of the problem
over the coming years.

Our overall approach is to specify a system model describing the
underlying rates and data models describing the relationship betwen the
rates and the available data, and then jointly infer all unknown
quantities. This type of hierarchical modelling is becoming increasingly
common in demography, epidemiology, ecology, and related disciplines
{[}e.g. 13,14--17{]}, though it is still uncommon in studies of obesity.
Distinctive features of our analysis include the diversity of the
information sources that we combine, the flexibility of our model of
prevalences, the integration of estimation and forecasting, and the
important role played by informative prior distributions.

We start the paper with a simple model, and then progressively add
extensions. The initial model is restricted to national-level trends,
uses a single data source, and has no provision for measurement error.
The first extension is to bring in two extra data sources and allow for
measurement error. The second extension is to use international
estimates of obesity trends to create informative priors, to reduce
uncertainty about rates of change. The third extension is to add region
and urban-rural residence. The fourth extension is to reformulate the
prior describing the accuracy of the administrative data, in response to
implausible patterns in the disaggregated results. We conclude the
analysis by probing some of our modelling assumptions. In the final
section of the paper, we argue that our methods are applicable to a wide
range of estimation and forecasting problems. Data and code to replicate
the analysis are available at
https://github.com/johnrbryant/bayescombwho and
https://github.com/johnrbryant/bayescomb.

\hypertarget{data}{%
\section{Data}\label{data}}

\hypertarget{national-health-examination-survey}{%
\subsection{National Health Examination
Survey}\label{national-health-examination-survey}}

We use data from the 1991, 1997, 2004, 2009, and 2014 rounds of the Thai
National Health Examination Survey (NHES). The survey is nationally
representative, with a complex design including stratification and
clustering. Interviews are conducted by local health personnel, who also
measure the height and weight of respondents {[}18,19{]}. We restrict
our analysis to respondents aged 2--17 years. The number of respondents
in the target ages varies from 661 in 2004 to 9,247 in 2009, with an
average of 6,058. Some rounds of the survey omit some ages within the
2--17 range.

\hypertarget{holistic-development-of-thai-children-survey}{%
\subsection{2001 Holistic Development of Thai Children
Survey}\label{holistic-development-of-thai-children-survey}}

The 2001 Holistic Development of Thai Children (HDTC) Survey collected
data from 17 provinces on topics related to family and childrearing,
including heights and weights of children {[}20{]}. We use data on
height and weight for 2,127 respondents aged 2--4 years.

\hypertarget{subsec:schools}{%
\subsection{Schools data}\label{subsec:schools}}

The Thai Office of Basic Education Commission collects data on the
health and socio-economic status of students at preschools, elementary
schools, and high schools receiving government subsidies. The data
collected includes heights and weights. Sample sizes are large, ranging
from 6,380,145 students aged 2--17 in 2019 to 6,883,478 in 2013. Using
information on the locations of schools, we construct a variable
distinguishing between eight regions of Thailand, and a variable
distinguishing urban areas from rural areas.

Coverage of the schools data is incomplete, with the number of children
in the dataset representing, on average, only 51\% of all children in
the corresponding age groups. Coverage is uneven across regions and age
groups, though children aged 2--9 in Bangkok have by far the lowest
coverage, with coverage rates of less than 5\%. Measurement of height
and weight is also uneven. Although teachers are supposed to take
measurements themselves, anecdotal evidence suggests that in some
schools teachers merely ask children their heights and weights. The
individual-level weight and height measurements also have some heaping
around values ending in 0 and 5.

\hypertarget{population-estimates}{%
\subsection{Population estimates}\label{population-estimates}}

To obtain population estimates for the period 1990--2010, we start with
counts by age, sex, region, and urban-rural residence from 1\% census
sample files for 1990, 2000, and 2010 {[}21{]}, and then interpolate
between census years using splines, with one spline for each combination
of age, sex, region, and urban-rural residence. For the period
2011--2019, we use 2010-base population projections from the Thai
National Economic and Social Development Board {[}22{]}.

\hypertarget{who-international-obesity-estimates}{%
\subsection{WHO international obesity
estimates}\label{who-international-obesity-estimates}}

We use WHO annual estimates of obesity prevalence among children aged
5--19 for 191 countries for the period 1990--2016, downloaded from the
WHO website {[}23{]}. The estimates are derived from many data sources,
and involve considerable imputation, interpolation, and smoothing. The
estimates distinguish between females and males, but not between age
groups. The WHO uses slightly different thresholds for height and weight
to define obesity than we do with the Thai data. All WHO estimates come
with confidence intervals.

\hypertarget{ethics-statement}{%
\subsection{Ethics statement}\label{ethics-statement}}

This study was approved by the Institute for Population and Social
Research Institutional Review Board (IPSR-IRB), at Mahidol University,
Thailand (COE. No.~2019/07-278).

\hypertarget{direct-estimates-of-obesity-prevalence}{%
\section{Direct estimates of obesity
prevalence}\label{direct-estimates-of-obesity-prevalence}}

We start with some direct estimates of obesity prevalence that do not
depend on statistical modelling. We measure obesity using body mass
index (BMI), which is defined as weight in kilograms divided by the
square of height in centimeters. A child is classified as obese if the
child's BMI exceeds the age-sex-specific cutoffs proposed by {[}24{]}.

Fig 1 shows direct estimates of prevalence based on the NHES, HDTC, and
schools data. We calculate point estimates and 95\% confidence intervals
from the survey data using standard design-based methods, as implemented
in the \emph{R} package \textbf{survey} {[}25{]}. The schools estimates
are simply the number of students who are obese divided by the number of
students in the schools dataset.

\begin{quote}
\textbf{Fig1. Direct estimates of national-level obesity prevalence by age, sex, and year from the NHES, HDTC, and schools data.} The dots represent point estimates from the NHES, and the x's represent point estimates from the HDTC; the vertical lines represent the associated 95\% confidence intervals. The HDTC data does not distinguish females and males, so the figure shows estimates for both sexes combined. The lines for the period 2013--2019 are prevalence estimates from schools data.
\end{quote}

The estimates in Fig 1 suggest that obesity prevalence is trending
upwards in all age groups, with the possible exception of children aged
2--4. There is also evidence that obesity is rising faster among males
than among females. The precision of the survey estimates varies across
different combinations of year and age, reflecting differences in sample
sizes. Schools data yields lower prevalence estimates than survey data,
except among the youngest children, though in most cases the trends are
in the same direction.

To measure differences between areas within Thailand, we rely entirely
on schools data. Subnational prevalence estimates for females are shown
in Fig 2. Results for males, which are similar to those for females, are
shown in the Supporting Information. There do appear to be small
differences between regions in obesity prevalence. Moreover, with one
exception, these differences appear to be consistent across age-sex
groups, with regions that have high prevalences in one age-sex group
having high prevalences in all the other age-sex groups. The exception
is Bangkok, which, according to the schools data, has relatively high
obesity below age 10, and relatively low obesity above age 10.

\begin{quote}
\textbf{Fig2. Direct estimates of obesity prevalence by age, region, and urban-rural residence for females, based on schools data.}
\end{quote}

\hypertarget{model-1-national-level-estimates-and-forecasts-using-only-nhes-data}{%
\section{Model 1: National-level estimates and forecasts, using only
NHES
data}\label{model-1-national-level-estimates-and-forecasts-using-only-nhes-data}}

\hypertarget{methods}{%
\subsection{Methods}\label{methods}}

Our first statistical model deals with obesity prevalence, by age and
sex, at the national level, using only the NHES to measure obesity.
Within each combination of age group \(a\), sex \(s\), and time \(t\),
we treat the number of children who are obese, \(y\), as a draw from a
binomial distribution with sample size \(n\) and probability \(\pi\).
Parameter \(\pi\) is the probability that a randomly-chosen child is
obese, and is the main quantity that we wish to infer.

Our model assumes that, within each combination of age, sex, and time,
respondents are a simple random sample of all Thai children. Respondents
in the NHES are not in fact sampled in this way. The complex survey
design implies that, even after conditioning on age, sex, and time, some
groups of Thai children have different probabilities of being included
in the NHES sample than others. Following {[}26{]} and {[}27{]}, we
account for the non-representativeness of the sample by fitting our
model to `effective' counts of respondents rather than to raw counts.
Effective counts equal raw counts scaled by factors that depend on
sample weights. An analysis that treats effective counts as if they come
from a simple random sample yields approximately the same means and
variances as a more complicated analysis that explicitly accounts for
the complex survey design. The Supporting Information gives the details.

Our model relating NHES data to prevalence \(\pi_{ast}\) is
\begin{equation}
  y_{ast}^{\text{EffN}} \sim \text{Binomial}(\pi_{ast}, n_{ast}^{\text{EffN}}), \label{eq:syslik1}
\end{equation} where the `EffN' superscript denotes effective counts
from the NHES. Prevalence \(\pi_{ast}\) is in turn modelled, on a logit
scale, as a draw from a normal distribution, the mean of which is the
product of row vector \(x_{ast}\) and column vector \(\beta\),
\begin{equation}
  \text{logit}(\pi_{ast}) \sim \text{N}( x_{ast} \beta, \sigma^2). \label{eq:sysprior1}
\end{equation} Transformation to the logit scale implies that values are
no longer bounded by 0 and 1. Vector \(\beta\) contains an intercept,
main effects for age, sex, and time, and interactions between age and
sex, age and time, and sex and time. Vector \(x_{ast}\), which consists
entirely of 1s and 0s, assigns the appropriate main effects and
interactions to each combination of age, sex, and time. The main effects
and interactions capture demographic regularities. For instance, the age
main effect captures the average age pattern across both sexes and all
times, and the age-sex interaction captures systematic differences
between the age patterns of females and males.

Estimates of the main effects and interactions can be stabilized by
adding to the model information about plausible ranges and patterns.
With a Bayesian model, this sort of additional information can be
encoded in prior distributions.

With the sex effect, we use a relatively simple prior distribution,
\begin{equation}
  \beta_s^{\text{sex}} \sim \text{N}(0, 1).
\end{equation} This prior captures the idea that, on a logit scale, we
might see female-male differences of -0.1 or 1.2, for instance, but not
-10 or 120. Bayesians refer to priors like this, which seek only to rule
out highly implausible values, rather than providing tight bounds on a
parameter, as `weakly informative' {[}10,28{]}.

The prior for the age effect is identical to the prior for the sex
effect. The prior for the intercept term has the same form as the prior
for the sex effect, but has a standard deviation of 10.

For time, we use a `damped linear trend' prior {[}29{]}, which is a
flexible version of a random walk with drift, \begin{align}
  \beta_t^{\text{time}} & \sim \text{N}(\alpha_t, \tau_{\beta}^2) \label{eq:sysmodtime1} \\
  \alpha_t & \sim \text{N}(\alpha_{t-1} + \delta_{t-1}, \tau_{\alpha}^2) \\
  \delta_t & \sim \text{N}(\phi \delta_{t-1}, \tau_{\delta}^2).
\end{align} Time effect \(\beta_t^{\text{time}}\) equals a level term
\(\alpha_t\) plus some random noise, the magnitude of which is governed
by \(\tau_{\beta}\). The level term at time \(t\) equals its value in
time \(t-1\), plus some random noise governed by \(\tau_{\alpha}\), plus
a drift term \(\delta_{t-1}\). The drift term captures any tendency for
upward or downward trends in obesity to persist over time. Empirical
studies of time series models have found that damping upward or downward
trends, rather than allowing them to continue indefinitely, tends to
improve forecast accuracy {[}30,31{]}. In our specification, damping is
controlled by parameter \(\phi\). Parameter \(\tau_{\delta}\) governs
the amount of random noise in \(\delta_t\).

To complete the prior for the time effect, we need to specify priors for
standard deviations \(\tau_{\beta}\), \(\tau_{\alpha}\), and
\(\tau_{\delta}\), and for \(\phi\). With each of the \(\tau\)s, we use
a half-normal distribution with a standard deviation of 1. A half-normal
distribution has the same shape as a normal distribution with mean zero,
but limited to non-negative values. We restrict \(\phi\) to the range
\([0.8, 1]\), and assume that \begin{equation}
  \frac{\phi - 0.8}{1 - 0.8} \sim \text{Beta}(2, 2).
\end{equation} All these priors qualify as weakly informative.

As we discuss below, the flexibility of the damped linear trend prior
makes it challenging to fit. However, less flexible versions of the
prior could potentially miss important features of the data. It is, for
instance, tempting to assume that, within each combination of age and
sex, rates of change are constant over time. Doing so would, however,
reduce our ability to detect turning points, and could produce forecasts
that were inappropriately confident.

The prior for the interaction between age and sex has the same structure
as the prior for the age and sex main effects. The prior for the
interaction between age and time is a variant on the prior for time, in
which each age group has its own linear trend model, but the variance
and damping parameters are shared across all age groups. The prior for
the interaction between sex and time is a linear trend model. Finally,
the \(\sigma\) in \eqref{eq:sysprior1} has the same half-normal prior as
the \(\tau\)s from the prior for time.

We estimate the model using function \texttt{estimateModel} from open
source \emph{R} package \textbf{demest}, available at
github.com/statisticsnz/R. Function \texttt{estimateModel} uses Markov
chain Monte Carlo methods, customised for demographic estimation. The
output from the modelling is a sample of values from the joint posterior
distribution for all unknowns quantities in the model {[}28{]}. We
summarise the joint posterior distribution by calculating medians, 50\%
credible intervals, and 95\% credible intervals of the distributions for
any unknown quantities we are interested in. The posterior medians serve
as point estimates.

The estimation process includes imputing values for years in which there
is no data. It is traditional to refer to this imputation process as
interpolation when the values being imputed lie in the past, and as
forecasting when the values lie in the future. In our model, however,
there is no strong distinction between imputation of past values and
imputation of future values, with exactly the same specification being
used for both.

\hypertarget{results}{%
\subsection{Results}\label{results}}

The top panel of Fig 3 shows results for prevalence \(\pi_{ast}\) from
our initial model. Although the median estimates and forecasts,
represented by the white lines, look reasonable, the 50\% and 95\%
credible intervals, represented by the dark and light bands, are
implausibly wide in years not covered by the NHES. From 2020 onwards,
the 95\% credible intervals essentially cover the entire range from 0 to
1.

\begin{quote}
\textbf{Fig3. Estimates and forecasts of national obesity prevalence from three models.} The top panel shows results for Model 1 using NHES data only, the middle panel shows results for Model 2 using NHES, HDTC, and schools data, and the bottom panel shows results for Model 3 with NHES, HDTC, and schools data plus WHO-based prior distributions for time terms.  The light bands represent 95\% credible intervals, the dark bands represent 50\% credible intervals, and the white lines represent medians. The black symbols represent direct estimates from Fig 1. The vertical axis for the top panel extends from 0 to 1, while the vertical axes for the other panels extend from 0 to 0.7.
\end{quote}

Examining estimates for higher-level parameters (not shown) indicates
that most of the uncertainty about prevalence \(\pi_{ast}\) comes from
uncertainty about time effects, and about age-time and sex-time
interactions. The parameter in the priors for time effects and
interactions that has the biggest influence on uncertainty is
\(\tau_{\delta}\), governing changes in the drift term. Higher values
for \(\tau_{\delta}\) imply bigger changes in drift term \(\delta_t\),
which, since the results compound over time, greatly increases the scope
for extreme outcomes.

The \(\tau_{\delta}\) parameters in the priors for the time, age-time,
and sex-time terms are estimated imprecisely. The 95\% credible interval
for the \(\tau_{\delta}\) in the time term, for instance, is (0.004,
0.563). Values towards the upper end of this range permit huge
year-on-year changes in \(\delta_t\).

The reason that \(\tau_{\delta}\) and the other parameters in the linear
trends priors are estimated imprecisely is that, with only five rounds,
the NHES provides limited information on change over time. Five time
points is large for a health survey, but tiny for a time series model.
For instance, of the 100,000 time series used in the M4 time series
competition -- a major empirical comparison of time series models -- the
shortest series had 15 time points {[}31{]}.

\hypertarget{model-2-adding-hdtc-and-schools-data}{%
\section{Model 2: Adding HDTC and schools
data}\label{model-2-adding-hdtc-and-schools-data}}

\hypertarget{methods-1}{%
\subsection{Methods}\label{methods-1}}

In our first extension, we expand our model to accommodate HDTC and
schools data. Fig 4 compares the structures of our first and second
models. Both models deal with counts of obese children, but the nature
of these counts differ. In the first model, the counts are of obese
children in the NHES, which are known with certainty. In the second
model, the counts are of obese children in all of Thailand, which are
unknown, and must be inferred. Even in the second model, however, total
numbers of children at risk of obesity, disaggregated by age and sex,
are treated as known.

\begin{quote}
\textbf{Fig4. The structure of our first and second models.} Our first model, on the left, allows for a single data source with sampling errors but not measurement errors. Our second model, on the right, allows for multiple data sources, all with sampling and measurement errors. Observed quantities are shaded gray; everything else is unobserved, and must be inferred.
\end{quote}

The counts of obese children in the second model are inferred from three
datasets. Each of these datasets provides imperfect and incomplete
measurements of obese children in all of Thailand. The NHES dataset in
our second model differs from the NHES dataset in our first model.
Rather than being a set of effective counts from the NHES survey, it is
a set of estimates for children in all of Thailand in each year of the
survey. These estimates are obtained from individual-level NHES data
using standard methods for complex survey data, as implemented in the
\textbf{survey} package. Similarly, the HDTC dataset consists of an
estimate for ages 2--4 in 2001 constructed from the raw individual-level
HDTC data. The schools dataset is constructed by multiplying
schools-based prevalence estimates like those in Fig 1 by population
estimates for the corresponding age, sex, and year.

The system model for our second overall model replaces
\eqref{eq:syslik1} from our first overall model with \begin{equation}
  y_{ast}^{\text{True}} \sim \text{Binomial}(\pi_{ast}, n_{ast}^{\text{True}}). \label{eq:syslik2}
\end{equation} In every other way, including all the prior
distributions, the system models for Model 1 and Model 2 are identical.

To construct our data model for the NHES, we rely on features of the
design of the survey, which is a common strategy in Bayesian analyses of
multiple data sources {[}8,e.g. 15{]}. The design of the survey implies
that the NHES estimates should be unbiased, and that errors in these
estimates should be approximately normally distributed. Moreover, the
standard deviations for these errors can be estimated through
design-based methods that exploit information about the survey including
sample weights.

Our data model for the NHES is \begin{equation}
  y_{ast}^{\text{EstN}} \sim \text{N}(y_{ast}^{\text{True}}, \kappa_{ast}^2), \label{eq:datamodnhes}
\end{equation} where the superscript `EstN' denotes `estimates derived
from the NHES.' We set the \(\kappa_{ast}\) equal to the standard
deviations that the \emph{R} package \textbf{survey} produces alongside
the estimates \(y_{ast}^{\text{True}}\). (Code for the design-based
calculations s included in the repository
https://github.com/johnrbryant/bayescomb.)

Values for \(y_{ast}^{\text{EstN}}\) are available for only some values
of \(y_{ast}^{\text{True}}\). For instance, no values for
\(y_{ast}^{\text{EstN}}\) are available for ages 2--4 and 5--9 in 2004,
or for any age groups in 2005--2008. The gaps in the data pose no
difficulties for estimation: the data model uses the values of
\(y_{ast}^{\text{True}}\) that it needs to predict the corresponding
values in the NHES dataset, and ignores the rest.

The data model for the HDTC estimates has the same structure to the NHES
model, except that the subscripts change from \(ast\) to \(at\), since
the HDTC data does not specify the sexes of the children. The fact that
the data lack a dimension contained in \(y_{ast}^{\text{True}}\) also
does not pose any difficulties for estimation: within the estimation
process, the sex dimension in \(y_{ast}^{\text{True}}\) is aggregated
away before the values are supplied to the data model.

The data model for the schools dataset is \begin{align}
  \log y_{ast}^{\text{EstS}} & \sim \text{N}(\log y_{ast}^{\text{True}} + \gamma_a, \varsigma_y^2) \label{eq:lik_school} \\
  \gamma_a & \sim \text{N}(0, \varsigma_{\gamma}^2).
\end{align} The log of the schools-based estimate equals the log of the
true count, plus an age-specific bias term \(\gamma_a\), plus random
noise. The use of logs implies that the data model is expressed in terms
of percentage errors, rather than absolute errors. We focus on
differences in biases across age groups because comparisons of direct
estimates, shown in Fig 1, suggests that these differences are large. As
illustrated in Fig 2 in the Supporting Information, dropping the
age-specific bias term from the model produces an implausible jump in
obesity prevalences for ages 2--4 when moving from NHES and THDS data to
schools data. In principle, we could extend the model to allow for
systematic differences across other dimensions besides age. However,
allowing too much flexibility in data models can make the overall model
difficult to fit.

The age-specific bias terms are drawn from a common distribution
centered on 0. Parameter \(\varsigma_y\) governs the amount of random
noise, and parameter \(\varsigma_{\gamma}\) governs variability in
age-specific bias. We use half-normal priors with standard deviations of
1 for \(\varsigma_y\) and \(\varsigma_{\gamma}\).

The likelihood from combining the three datasets has the form
\begin{equation*}
  \begin{split}
  p(\text{data} | \text{true obesity counts}) & = p(\text{NHES data} | \text{true obesity counts}) \\
      & \quad \quad \times p(\text{HDTC data} | \text{true obesity counts}) \\
      & \quad \quad \times p(\text{Schools data} | \text{true obesity counts}).
   \end{split}
\end{equation*} The (unobserved) true obesity counts appear multiple
times in the likelihood. The repetition does not, however, cause any
problems. It is analogous to having the same regression coefficients
occur multiple times in the likelihood for a regression model.

\hypertarget{results-1}{%
\subsection{Results}\label{results-1}}

The middle panel of Fig 3 shows results from Model 2. Adding the extra
data sources dramatically reduces uncertainty compared with the first
model. (Note that the top and middle panels of Fig 3 use different
vertical scales.) The reduction in overall uncertainty results, in large
part, from a reduction in uncertainty about the \(\tau_{\delta}\)
parameters in the main effect and interactions involving time. For
instance, the 95\% credible interval for \(\tau_{\delta}\) in the main
effect for time in the second model is (0.001, 0.063), compared with
(0.004, 0.563) in first model.

With uncertainty at more reasonable levels, differences in prevalences
by age and sex become more apparent. The estimates for ages 2--4 imply
that prevalences rose only slightly during the 1990s and 2010s. The
relatively flat forecasts for ages 2--4 extrapolate these slow changes
into the future. The estimates have more pronounced upward trends among
other age groups, particularly for males. Prevalences rise particularly
quickly among males aged 15--17 during the 2010s and 2020s. These
changes appear to reflect the rapid growth (from a low base) in obesity
among males aged 15--17 in the schools data.

The results for Model 2 are consistent with the idea that the
relationship between actual prevalence and prevalence in the schools
data varies with age. School-based prevalences more or less match the
modelled estimates at ages 2--4, but are substantially lower among the
other age groups.

\hypertarget{sec:model3}{%
\section{Model 3: Adding WHO-based priors for time
effects}\label{sec:model3}}

\hypertarget{methods-2}{%
\subsection{Methods}\label{methods-2}}

Although our second overall model gives more sensible estimates of
uncertainty than the first, the second model still appears to overstate
the probability of sudden shifts in prevalences, judging by the wide
95\% credible intervals for 2030. This suggests that, even with the
addition of the HDTC and schools data, the parts of the model dealing
with change over time could benefit from more data. In the absence of
more Thai data on change over time, we turn to data for other
countries---specifically, the WHO estimates of obesity trends in 191
countries.

Our strategy is to fit a simple model covering all 191 countries, and
extract from the model parameter values capturing typical year-to-year
variability. We use these parameters to formulate informative priors for
parameters governing change over time in our Thai model. A prior is
informative if it places relatively tight bounds on a parameter, and
thus potentially has a substantial effect on final estimates.

We convert the WHO point estimates and 95\% confidence intervals into
effective counts of children, as described in the Supporting
Information. We then fit the model \begin{align}
  y_{cst}^{\text{WHO}} & \sim \text{Binomial}(\pi_{cst}^{\text{WHO}}, n_{cst}^{\text{WHO}}) \\
  \text{logit}(\pi_{cst}^{\text{WHO}}) & \sim \text{N}(x_{cst}^{\text{WHO}} \beta^{\text{WHO}}, \sigma_{\text{WHO}}^2),
\end{align} where \(c\) denotes country, \(s\) sex, and \(t\) time.
Vector \(\beta^{\text{WHO}}\) contains a country effect, a sex effect, a
country-sex interaction, and a time effect. The sex effect has a normal
prior with a standard deviation of 1. The prior for the country effect
is \(\beta_c^{\text{ctry}} \sim \text{N}(0, \tau_{\text{ctry}}^2)\),
with \(\tau_{\text{ctry}}\) having a half-normal prior with standard
deviation 1. The prior for the country-sex interaction is identical to
the prior for the country effect, except that the \(c\) subscript is
replaced by a \(cs\) subscript. The time effect has a linear trend
prior, with exactly the same specification as the prior for time effects
and interactions in our Thai models.

Having fitted the model to the WHO data, we discard all results except
those for the time effect. In particular, we retain samples from the
posterior distributions for time effect parameters \(\tau_{\beta}\),
\(\tau_{\alpha}\), \(\tau_{\delta}\), and \(\phi\). We use these samples
to create informative priors for the corresponding parameters in the
time effect, age-time interaction, and sex-time interaction in our
models for Thailand.

Based on the sample from the posterior distribution for \(\tau_{\beta}\)
in the WHO model, for instance, it appears that the posterior
distribution can be closely approximated by the half-normal distribution
\(\text{N}^+(0, 0.0054^2)\). The distribution
\(\text{N}^+(0, 0.0054^2)\) can be used as an informative prior for
\(\tau_{\beta}\) in the time effect, age-time interaction, and sex-time
interaction in the Thai model. Informative priors for \(\tau_{\alpha}\),
\(\tau_{\delta}\), and \(\phi\) in the time effect, age-time
interaction, and sex-time interaction can be constructed in similar
ways. Details are provided in the Supporting Information.

An important feature of our use of the WHO data is that we only extract
information about rates of change. The parameters \(\tau_{\beta}\),
\(\tau_{\alpha}\), \(\tau_{\delta}\), and \(\phi\) all deal with
year-to-year variation, and not with absolute levels. Even with the
WHO-based priors for time effects, our Thai models continue to rely on
Thai data to determine absolute levels. We assume that even if
differences between the WHO and Thai data in the age groups covered and
in the thresholds used to define obesity make it difficult to pool
information about absolute levels, they do not affect our ability to
pool information about rates of change.

The main advantage of the WHO estimates is that they summarise the
experiences of virtually every country in the world. The main
disadvantage, for our purposes, is that many country-specific time
series have been subject to substantial smoothing, imputation, and
interpolation. The resulting time series may understate actual
year-to-year variation. To allow for this possibility, we modify the
prior distributions for \(\tau_{\beta}\), \(\tau_{\alpha}\),
\(\tau_{\delta}\), and \(\phi\) so that they have approximately twice
the variance of the original WHO-based versions, and use these modified
priors, rather than the original informative priors, in models 3--5.
(The Supporting Information includes a description of the modification
process.) The decision of how much to scale the variances of the priors
is inevitably somewhat arbitrary. We test the sensitivity of our results
to alternative choices later in the paper.

\hypertarget{results-2}{%
\subsection{Results}\label{results-2}}

The bottom panel of Fig 3 shows results from Model 3. Comparison of the
middle and bottom panels of Fig 3 suggests that the use of WHO-based
priors has virtually no effect on historical estimates. Use of the
WHO-based priors does, however, lead to somewhat narrower credible
intervals for 2030. Exploiting the information about plausible annual
variation contained in the WHO estimates leads to a modest increase in
the precision of the forecasts.

\hypertarget{model-4-disaggregating-by-region-and-by-urban-rural-residence}{%
\section{Model 4: Disaggregating by region and by urban-rural
residence}\label{model-4-disaggregating-by-region-and-by-urban-rural-residence}}

\hypertarget{methods-3}{%
\subsection{Methods}\label{methods-3}}

In Model 4, we disaggregate the estimates and forecasts by region and by
urbal-rural residence. Equation \eqref{eq:syslik2} in the system model
is replaced by \begin{equation}
  y_{asrut}^{\text{True}} \sim \text{Binomial}(\pi_{asrut}, n_{asrut}^{\text{True}}), \label{eq:likmod4}
\end{equation} which is identical to \eqref{eq:syslik2}, except for the
addition of the \(r\) subscript denoting region and the \(u\) subscript
denoting urban-rural residence.

The main effects used to predict \(\text{logit}(\pi_{asrut})\) in Model
4 are the same as those in Model 3, along with a main effect for region
and a main effect for urban-rural residence. The prior distributions for
region and urban-rural residence have the same format as the prior
distribution for country in the WHO model.

The data models for the NHES and HDTC are unchanged. Within the
estimation process, the region and urban-rural dimensions of
\(y_{asrut}^{\text{True}}\) are aggregated away before the obesity
counts are supplied to the data models.

The data model for the schools data is expanded to include \(r\) and
\(u\) subscripts, \begin{align}
  \log y_{asrut}^{\text{EstS}} & \sim \text{N}(\log y_{asrut}^{\text{True}} + \gamma_a, \varsigma_y^2) \label{eq:likschools4} \\
  \gamma_a & \sim \text{N}(0, \varsigma_{\gamma}^2).
\end{align} The revised data model for schools, like the original data
model for schools, only allows biases to vary by age. Biases may in fact
vary by region or by urban-rural residence, but with only one source of
data on subnational variation in obesity prevalence, there is limited
scope for distinguishing between subnational variation in coverage and
subnational variation in actual prevalence.

\hypertarget{results-3}{%
\subsection{Results}\label{results-3}}

The top panel of Fig 5 shows results from Model 4. To save space, Fig 5
only includes results for females in urban areas; results for males and
for rural areas are shown in the Supporting Information. Each age group
in Fig 5 follows a similar trajectory to the corresponding
national-level age group in Fig 3, though these trajectories vary
slightly between regions, reflecting patterns in the raw data in Fig 2.
The Northeast region, for instance, has relatively low prevalences in
Fig 5, reflecting the relatively low direct estimates of prevalence
shown in Fig 2. Model 4 seems, in many ways, to be giving sensible
results.

\begin{quote}
\textbf{Fig5. Estimates and forecasts of obesity prevalence for females in urban areas, from models 4 and 5.} The top panel shows results from our first subnational model (Model 4), and the bottom panel shows results from our revised model (Model 5).  The light bands represent 95\% credible intervals, the dark bands represent 50\% credible intervals, and the white lines represent medians. The black symbols represent direct school-based estimates from Fig 2.
\end{quote}

One implausible feature of the results from Model 4, however, is the
sudden departures from long-term trends that occur in many series in
response to the schools data. In Bangkok, for instance, estimates for
ages 2--4 and 5--9 shift sharply upwards during the years where the
schools data are available, while estimates for ages 10--14 and 15--17
shift sharply downwards. In regions such as Bangkok and Central
Thailand, the estimates closely track year-to-year changes in the
schools data.

The sudden departures from long-term trends and the close tracking of
the schools data evident in the top panel of Fig 5 reflect the fitted
values for parameters \(\sigma\) and \(\varsigma_y\) in Model 4.
Parameter \(\sigma\), from the subnational equivalent of
\eqref{eq:sysprior1}, governs the amount of idiosyncratic variation in
true obesity prevalence. Parameter \(\varsigma_y\), from
\eqref{eq:likschools4}, governs the accuracy of the schools data after
accounting for age-specific biases. The point estimate for \(\sigma\) is
0.132 in Model 4, compared with 0.013 in Model 3, while the point
estimate for \(\varsigma_y\) is 0.023 in Model 4, compared with 0.010 in
Model 3. These numbers imply that subnational prevalence rates are much
less stable than national rates, while subnational school-based measures
of these rates are only slightly less accurate than national
school-based measures.

We would, instead, expect Model 4 to have slightly higher values for
\(\sigma\) than Model 3, and much higher values for \(\varsigma_y\).
Some increase in idiosyncratic variation in prevalence rates when moving
from the national level to the subnational level is plausible, but a
10-fold increase is not. Conversely, given the low coverage rates in
regions such as Bangkok, and the complications introduced by student
migration, subnational school-based estimates of obesity prevalence
could be expected to be substantially less reliable than those
national-level estimates. Some deficiency in Model 4 seems to be leading
it to misrepresent variation in true prevalences and misrepresent the
accuracy of the schools data.

\hypertarget{sec:model5}{%
\section{Model 5: Adjusting the data model for the schools
data}\label{sec:model5}}

\hypertarget{methods-4}{%
\subsection{Methods}\label{methods-4}}

We obtain our final model, Model 5, by adjusting the data model for the
schools data. In Model 4, and in earlier models, we use the same
half-normal prior for \(\varsigma_y\) that we use for scale parameters
in the system model. A half-normal prior favours values near zero. In
the system model, using a prior for scale parameters that favours values
near zero damps down variation in parameters, such as the \(\beta\) or
\(\pi\), that are governed by the scale parameter. Damping down
variation in these parameters provides robustness to noise in the data.
In the data model for schools, however, using a prior for
\(\varsigma_y\) that favours values near zero reduce robustness, instead
of increasing it. Values of \(\varsigma_y\) near zero imply that (after
accounting for age-specific biases) most observed variation in schools
data reflects real variation in underlying prevalences. When
\(\varsigma_y\) is low, variation in the data is propagated through to
estimates of underlying prevalences. This is not appropriate behavior
when, as with the subnational schools data, the data is likely to have
substantial measurement error.

We replace the half-normal prior for \(\varsigma_y\) with one that no
longer favours values near zero. We switch from a half-normal prior to a
scaled inverse-\(\chi^2\) distribution, which has a mode away from zero
and a long right tail. For mathematical convenience, we apply the prior
to \(\varsigma_y^2\) rather than to \(\varsigma_y\) itself. A scaled
inverse-\(\chi^2\) distribution has a degrees-of-freedom parameter
which, roughly speaking, controls the dispersion of the distribution,
and a scale-squared parameter which controls the mean. We derive values
for these parameters that try to reflect the plausible range of values
for \(\varsigma_y^2\).

Let \(p_{asrut}\) be a direct estimate of obesity prevalence from the
schools data, as depicted in Fig 2. Our data model for the schools data
implies that \begin{equation}
  \log p_{asrut} = \log \pi_{asrut} + \gamma_a + u_{asrut}. \label{eq:accuracy1}
\end{equation} where \(\pi_{asrut}\) is the true prevalence, and
\(u_{asrut}\) has a normal distribution with mean 0 and variance
\(\varsigma_y^2\). We approximate \(\log \pi_{asrut}\) with
\(z_{asrut} \eta\), where \(\eta\) contains the same main effects and
interactions as our subnational system model and \(v_{asrut}\) is a
vector of 1s and 0s. Substituting into \eqref{eq:accuracy1} gives
\begin{align}
  \log p_{asrut} & = z_{asrut} \eta + \gamma_a + u_{asrut} \\
  & = z_{asrut}' \eta' + u_{asrut}, \label{eq:accuracy2}
\end{align} where \(z_{asrut}' \eta' = = z_{asrut} \eta + \gamma_a\). We
fit the model of \eqref{eq:accuracy2} using least squares, as
implemented in function \texttt{lm} in \emph{R} package \textbf{stats},
obtaining a point estimate for \(\varsigma_y^2\) of 0.016.

We set the degrees-of-freedom parameter for the scaled
inverse-\(\chi^2\) prior distribution to 30 and the scale-squared
parameter to 0.015. These values imply that there is a 95\% chance that
the true value of \(\varsigma_y^2\) is between 0.010 and 0.027.

\hypertarget{results-4}{%
\subsection{Results}\label{results-4}}

Results from our revised model, Model 5, for females in urban areas, are
shown in the lower panel of Fig 5. Results for males and for rural areas
are included in the Supporting Information. The estimate from Model 5
are smoother than those from Model 4. Estimated prevalences still
respond to changes in the schools data, but not nearly as closely, and
without the dramatic upward and downward shifts.

Replacing the prior for \(\varsigma_y\) leads, as expected, to lower
estimates for \(\sigma\) and higher estimates for \(\varsigma_y\). The
new point estimate for \(\sigma\) is 0.073, compared to 0.132 in Model
4, and the new point estimate for \(\varsigma_y\) is 0.105, compared to
0.023 in Model 4.

\hypertarget{checking-the-model}{%
\section{Checking the model}\label{checking-the-model}}

An essential part of any modelling workflow is to assess how sensitive
the results are to alternative possible specifications, and to consider
whether the model has captured all the substantively-important features
of the system under study {[}32{]}. A full suite of model checking would
require more space than is available here, but we present two
illustrative examples.

\hypertarget{subsec:sensitivity}{%
\subsection{Sensitivity to priors for time main effects and
interactions}\label{subsec:sensitivity}}

As discussed above, we suspect that the WHO country-level estimates
understate annual variability in obesity prevalence, but are unsure by
how much. In Models 3--5, we use priors that, roughly speaking, entail
twice as much annual variability than is implied by the original WHO
estimates. Here we investigate how the results vary with alternative
choices. To gain insights into the way that the priors interact with
other parts of the model, we show results for each of the broader model
classes considered in the paper.

Each panel of Fig 6 shows national-level estimates and forecasts,
aggregating over sex and age. Each column of panels shows how result
vary with the choice of prior. The priors are ordered from weakest to
strongest. The first column is our original weakly-informative
half-normal prior. The fourth column is the prior obtained by using the
unadjusted values obtained from the WHO model. The third column is the
prior obtained by inflating variances by a factor of 2. The second
column is the prior obtained by inflating variances by a factor of 4.
Each row of the figure shows results for a different overall model
specification, starting with the national-level NHES-only model in row
1, and finishing with the subnational model with the revised data model
for schools in row 4. Panel (1, 1) in the figure corresponds to Model 1;
panel (2, 1) to Model 2; panel (2, 3) to Model 3; panel(3, 3) to Model
4; and panel (4, 3) to Model 5.

\begin{quote}
\textbf{Fig6. Comparison of national-level estimates and forecasts of obesity prevalence from different combinations of model type and priors for variance terms in time main effects and interactions.} The estimates and forecasts aggregate over age, sex, region, and urban-rural residence. Each row of panels represents one model type, and each column represents one prior.
\end{quote}

Replacing the default weakly informative time prior with a WHO-based
prior reduces forecast uncertainty under all four model specifications,
though it has the biggest effect in the national-level NHES-only model.
With the exception of the national-level NHES-only model, stronger
WHO-based priors lead to more linear forecasts, with less tapering-off
in growth rates towards the end of the period. When comparing across
WHO-based priors, however, the choice of prior has only a minor effect
on the shape or uncertainty of estimates in years where there is data.

\hypertarget{replicate-data-checks}{%
\subsection{Replicate data checks}\label{replicate-data-checks}}

A Bayesian statistical model treats the observed data as a draw from a
probability distribution. If the model is a good representation of the
process being studied, then it should be possible to use the model to
randomly generate hypothetical datasets that look similar to the real
dataset. Conversely, if hypothetical datasets generated by the model are
systematically different from the real dataset, then the model may be
deficient. Bayesians call the technique of comparing hypothetical
datasets with the real dataset ``replicate data checks'' {[}28,32{]}.

Replicate data checks should be targeted at possible weak points in the
model. One possible weak point of Model 5 is that the prior model for
the prevalences \(\pi_{asrut}\) assumes a common time trend across all
regions, and across urban and rural areas. Any geographical variation in
the pace at which measured obesity is increasing is assumed to arise
from random variation in probabilities of being obese, random variation
in obesity counts given probabilities, and random variation in measured
counts given true counts. It is possible that this approach implies too
much uniformity across regions and across urban and rural areas.

We use replicate data checks to assess the ability of Model 5 to
generate geographical variation in rates of change. We focus on the
schools dataset, since the schools dataset is the only one that is
disaggregated by geography. Each replicate dataset is generated as
follows:

\begin{enumerate}
\def\labelenumi{\arabic{enumi}.}
\tightlist
\item
  Randomly select a draw \(k\) from the sample from the joint posterior
  distribution from Model 5, and obtain values \(\beta^{(k)}\),
  \(\sigma^{(k)}\), \(\gamma_a^{(k)}\) and \(\varsigma_y^{(k)}\) .
\item
  Plug \(\beta^{(k)}\) and \(\sigma^{(k)}\) into the prior model for
  prevalences, and generate values \(\pi_{asrut}^{(k)}\).
\item
  Plug the \(\pi_{asrut}^{(k)}\) into \eqref{eq:likmod4}, and generate
  values \(y_{asrut}^{(k)}\) for true numbers of obese children.
\item
  Plug \(y_{asrut}^{(k)}\) into \eqref{eq:likschools4}, and generate
  values \(y_{asrut}^{\text{EstS}(k)}\) for school-based estimates of
  obese children.
\end{enumerate}

Repeating this process \(K\) types yields \(K\) replicate datasets. The
observed and replicate datasets are too big to work with directly, so we
calculate summary measures of geographical variation in rates of change
for each dataset and compare the summary measures instead. For each
combination of age, sex, region, and urban-rural residence in each
dataset, we fit a straight line through values for
\(y_{asrut}^{\text{EstS}}\) versus time. The slopes of these lines are
our summary measures.

Fig 7 shows the results from these calculations. The first column of
panels gives results for the observed dataset, and the remaining columns
give results for the replicate datasets. Each dot represents a slope
estimate. To save space, the figure only shows results for females.

\begin{quote}
\textbf{Fig7. Using replicate datasets to assess the ability of the model to capture geographical variability in rates of change for obesity.} Each columns shows results from one dataset. Each dot represents the slope from a regression of numbers of obese children against time. The figure shows results for females only.
\end{quote}

What we are looking for in Fig 7 is evidence of systematic differences
between the observed dataset and the replicate datasets. Given that all
datasets, including the observed one, have an element of randomness, we
are not looking for complete agreement, but instead for whether the
observed dataset is in some sense an outlier.

Inspection of Fig 7 suggests that the observed dataset is not an
outlier. The observed dataset contains about the same amount of
geographical variation in rates of change as the replicate datasets do.
The use of a common time trend across regions and urban-rural residents
appears to be a reasonable approximation.

\hypertarget{discussion}{%
\section{Discussion}\label{discussion}}

In this paper, we have developed disaggregated estimates and forecasts
of obesity among Thai children. We have found that trends differ
substantially by age: prevalences for children under 5 have shown little
change since the early 1990s, while prevalences for children aged 15--17
have, in recent years, increased sharply, with a strong prospect of
reaching rates of 25\% or more by 2030. Prevalences have varied across
different parts of the country, though forecasted prevalences show
substantial overlap. Supplementing the National Health Examination
Survey data with additional survey and administrative data reduces
uncertainty considerably, and exploiting international data on trends in
obesity prevalence reduces uncertainty further. Even with the extra
information, however, the forecasts are still sensitive to alternative
assumptions about the variability in rates of change. Our ability to
measure geographical variation is also constrained by the fact that our
only source of information on geographical variation is subject to
measurement error. Further improvements in the estimates and forecasts
are likely to require additional data.

The apparent divergence in trends for children below and above age 5 is
new to the study of obesity in Thailand. Most children in Thailand begin
attending some form of schooling around this age, which suggests that
there may be something about the school environment that leads to higher
rates of obesity. In addition, the forecasts that obesity will become
common across all regions of the country, and in both urban and rural
areas, suggests that policies to address rising obesity rates need to be
implemented nationally, and not just in places where obesity is already
high.

The analytical framework used in this paper can accommodate a wide range
of applications involving disaggregated estimation and forecasting. The
system model, the set of unobserved counts, and the data models can all
be customised to the problem at hand. The resulting models can be
complex. However, the individual components from which the models are
composed are often simple and intuitive, and the models can be built up
piece by piece. Combining the assessment of data quality, the estimation
of underlying rates, and the forecasting of future values into a single
framework allows us to capture uncertainties in a unified and
internally-consistent way.

The ability to combine multiple data sources allows frequent updating of
estimates and forecast. In the case of Thai obesity, for instance,
estimates and forecasts of obesity can be revised each time new schools
data becomes available, rather than waiting for a new round of the
National Health and Examination Survey -- though regular updates of the
NHES data are still important to adjust for errors in the schools data.

A dictinctively Bayesian part of our analysis is the use of informative
prior distributions. We use an informative prior distribution to
describe the plausible range for annual change in obesity rates, and use
an informative prior distribution to describe the likely size of
measurement errors in the schools data. In both cases, the informative
prior distributions are grounded in quantitative analyses, and are
transparent and replicable. Incorporating informative priors into the
model increases the plausibility and precision of the resulting
estimates and forecasts.

\hypertarget{supporting-information}{%
\section{Supporting Information}\label{supporting-information}}

\begin{quote}
\textbf{Supporting Information -- contains all the supporting figures and text.}
\end{quote}

\hypertarget{acknowledgements}{%
\section{Acknowledgements}\label{acknowledgements}}

We would like to thank the National Health Examination Survey, the
Office of the Basic Education Commission of Thailand, the Holistic
Development of Thai Children Project, the National Statistics Office of
Thailand, the Thai National Economic and Social Development Board, and
the World Health Organization for data used in this paper. We are also
grateful to colleagues who provided feedback on earlier versions of the
paper.

\hypertarget{references}{%
\section*{References}\label{references}}
\addcontentsline{toc}{section}{References}

\hypertarget{refs}{}
\begin{CSLReferences}{0}{0}
\leavevmode\hypertarget{ref-tatem2017worldpop}{}%
\CSLLeftMargin{1. }
\CSLRightInline{Tatem AJ. WorldPop, open data for spatial demography.
Scientific data. Nature Publishing Group; 2017;4: 1--4. }

\leavevmode\hypertarget{ref-murray2020five}{}%
\CSLLeftMargin{2. }
\CSLRightInline{Murray CJ, Abbafati C, Abbas KM, Abbasi M,
Abbasi-Kangevari M, Abd-Allah F, et al. Five insights from the {G}lobal
{B}urden of {D}isease {S}tudy 2019. The Lancet. Elsevier; 2020;396:
1135--1159. }

\leavevmode\hypertarget{ref-hand2018statistical}{}%
\CSLLeftMargin{3. }
\CSLRightInline{Hand DJ. Statistical challenges of administrative and
transaction data. Journal of the Royal Statistical Society: Series A
(Statistics in Society). 2018;181: 555--605. }

\leavevmode\hypertarget{ref-gregory2010modelling}{}%
\CSLLeftMargin{4. }
\CSLRightInline{Gregory IN, Marti-Henneberg J, Tapiador FJ. Modelling
long-term pan-{E}uropean population change from 1870 to 2000 by using
geographical information systems. Journal of the Royal Statistical
Society: Series A (Statistics in Society). 2010;173: 31--50. }

\leavevmode\hypertarget{ref-hyndman2021forecasting}{}%
\CSLLeftMargin{5. }
\CSLRightInline{Hyndman RJ, Athanasopoulos G. Forecasting: Principles
and practice. Third edition {[}Internet{]}. OTexts; 2021. Available:
\url{https://otexts.com/fpp3/}}

\leavevmode\hypertarget{ref-lohr2017combining}{}%
\CSLLeftMargin{6. }
\CSLRightInline{Lohr SL, Raghunathan TE. Combining survey data with
other data sources. Statistical Science. 2017;32: 293--312. }

\leavevmode\hypertarget{ref-de2020multi}{}%
\CSLLeftMargin{7. }
\CSLRightInline{Waal T de, Delden A van, Scholtus S. Multi-source
statistics: Basic situations and methods. International Statistical
Review. 2020;88: 203--228. }

\leavevmode\hypertarget{ref-li2019changes}{}%
\CSLLeftMargin{8. }
\CSLRightInline{Li Z, Hsiao Y, Godwin J, Martin BD, Wakefield J, Clark
SJ, et al. Changes in the spatial distribution of the under-five
mortality rate: Small-area analysis of 122 {DHS} surveys in 262
subregions of 35 countries in {A}frica. PloS One. Public Library of
Science San Francisco, CA USA; 2019;14: e0210645. }

\leavevmode\hypertarget{ref-rubin1984bayesianly}{}%
\CSLLeftMargin{9. }
\CSLRightInline{Rubin DB. Bayesianly justifiable and relevant frequency
calculations for the applied statistician. The Annals of Statistics.
1984; 1151--1172. }

\leavevmode\hypertarget{ref-van2021bayesian}{}%
\CSLLeftMargin{10. }
\CSLRightInline{Schoot R van de, Depaoli S, King R, Kramer B, Märtens K,
Tadesse MG, et al. Bayesian statistics and modelling. Nature Reviews
Methods Primers. 2021;1: 1--26. }

\leavevmode\hypertarget{ref-aekplakorn2014prevalence}{}%
\CSLLeftMargin{11. }
\CSLRightInline{Aekplakorn W, Inthawong R, Kessomboon P, Sangthong R,
Chariyalertsak S, Putwatana P, et al. Prevalence and trends of obesity
and association with socioeconomic status in {T}hai adults: {N}ational
{H}ealth {E}xamination {S}urveys, 1991--2009. Journal of Obesity. 2014;
Available: \url{http://dx.doi.org/10.1155/2014/410259}}

\leavevmode\hypertarget{ref-bhurosy2014overweight}{}%
\CSLLeftMargin{12. }
\CSLRightInline{Bhurosy T, Jeewon R. Overweight and obesity epidemic in
developing countries: A problem with diet, physical activity, or
socieconomic status? The Scientific World Journal. 2014;2014.
doi:\url{http://dx.doi.org/10/1155/2014/964236}}

\leavevmode\hypertarget{ref-king2009bayesian}{}%
\CSLLeftMargin{13. }
\CSLRightInline{King R, Morgan B, Gimenez O, Brooks S. Bayesian analysis
for population ecology. CRC Press; 2009. }

\leavevmode\hypertarget{ref-raymer2013integrated}{}%
\CSLLeftMargin{14. }
\CSLRightInline{Raymer J, Wiśniowski A, Forster JJ, Smith PW, Bijak J.
Integrated modeling of {E}uropean migration. Journal of the American
Statistical Association. 2013;108: 801--819. }

\leavevmode\hypertarget{ref-alkema2016global}{}%
\CSLLeftMargin{15. }
\CSLRightInline{Alkema L, Chou D, Hogan D, Zhang S, Moller A-B, Gemmill
A, et al. Global, regional, and national levels and trends in maternal
mortality between 1990 and 2015, with scenario-based projections to
2030: A systematic analysis by the {UN} {M}aternal {M}ortality
{E}stimation {I}nter-{A}gency {G}roup. The Lancet. 2016;387: 462--474. }

\leavevmode\hypertarget{ref-regehr2018integrated}{}%
\CSLLeftMargin{16. }
\CSLRightInline{Regehr EV, Hostetter NJ, Wilson RR, Rode KD, Martin MS,
Converse SJ. Integrated population modeling provides the first empirical
estimates of vital rates and abundance for polar bears in the {C}hukchi
{S}ea. Scientific reports. 2018;8: 1--12. }

\leavevmode\hypertarget{ref-flaxman2020estimating}{}%
\CSLLeftMargin{17. }
\CSLRightInline{Flaxman S, Mishra S, Gandy A, Unwin HJT, Mellan TA,
Coupland H, et al. Estimating the effects of non-pharmaceutical
interventions on {COVID}-19 in {E}urope. Nature. Nature Publishing
Group; 2020;584: 257--261. }

\leavevmode\hypertarget{ref-aekplakorn2016report}{}%
\CSLLeftMargin{18. }
\CSLRightInline{Aekplakorn W, Pakcharoen H, Thaikla K, Satheannoppakao W
and. Report of the fifth national health examination survey 2014 (NHES
v) {[}Internet{]}. Health Systems Research Institute; 2016. Available:
\url{https://www.hsri.or.th/researcher/research/new-release/detail/7711}}

\leavevmode\hypertarget{ref-aekplakorn2018prevalence}{}%
\CSLLeftMargin{19. }
\CSLRightInline{Aekplakorn W, Chariyalertsak S, Kessomboon P,
Assanangkornchai S, Taneepanichskul S, Putwatana P. Prevalence of
diabetes and relationship with socioeconomic status in the {T}hai
population: {N}ational {H}ealth {E}xamination {S}urvey, 2004--2014.
Journal of Diabetes Research. 2018; Available:
\url{https://doi.org/10.1155/2018/1654530}}

\leavevmode\hypertarget{ref-mosuwan2004holistic}{}%
\CSLLeftMargin{20. }
\CSLRightInline{Mo-suwan L. Holistic development of thai children:
{F}amily and child rearing, 2001 (in {T}hai). Thailand Research Fund;
2003. }

\leavevmode\hypertarget{ref-nso1990population}{}%
\CSLLeftMargin{21. }
\CSLRightInline{National Statistics Office of Thailand. Population and
housing census {[}Internet{]}. National Statistics Office; 1990-\/-2010.
Available: \url{http://web.nso.go.th/en/census/poph/cen_poph.htm}}

\leavevmode\hypertarget{ref-nesdb2019population}{}%
\CSLLeftMargin{22. }
\CSLRightInline{National Economic and Social Development Board.
Population projections for {T}hailand 2010-2040: (The 2019 revision).
Office of the National Economic; Social Development Board; 2019. }

\leavevmode\hypertarget{ref-global2019prevalence}{}%
\CSLLeftMargin{23. }
\CSLRightInline{Global Health Observatory. Prevalence of obesity among
children and adolescents, BMI\textgreater+2 standard deviation above the
median, crude: Estimates by country, among children aged 5-19 years. WHO
website. {[}Internet{]}. 2019. Available:
\url{https://apps.who.int/gho/data/view.main.BMIPLUS2C05-19v}}

\leavevmode\hypertarget{ref-cole2000establishing}{}%
\CSLLeftMargin{24. }
\CSLRightInline{Cole TJ, Bellizzi MC, Flegal KM, Dietz WH. Establishing
a standard definition for child overweight and obesity worldwide:
International survey. British Medical Journal. 2000;320: 1240. }

\leavevmode\hypertarget{ref-lumley2004analysis}{}%
\CSLLeftMargin{25. }
\CSLRightInline{Lumley T. Analysis of complex survey samples. Journal of
Statistical Software. 2004;9: 1--19. }

\leavevmode\hypertarget{ref-ghitza2013deep}{}%
\CSLLeftMargin{26. }
\CSLRightInline{Ghitza Y, Gelman A. Deep interactions with {MRP}:
Election turnout and voting patterns among small electoral subgroups.
American Journal of Political Science. 2013;57: 762--776. }

\leavevmode\hypertarget{ref-chen2014use}{}%
\CSLLeftMargin{27. }
\CSLRightInline{Chen C, Wakefield J, Lumley T. The use of sampling
weights in {B}ayesian hierarchical models for small area estimation.
Spatial and Spatio-Temporal Epidemiology. 2014;11: 33--43. }

\leavevmode\hypertarget{ref-gelman2014bayesian}{}%
\CSLLeftMargin{28. }
\CSLRightInline{Gelman A, Carlin JB, Stern HS, Dunson DB and, Vehtari A,
Rubin DB. Bayesian data analysis. Third edition. Chapman \& Hall/CRC;
2014. }

\leavevmode\hypertarget{ref-prado2010time}{}%
\CSLLeftMargin{29. }
\CSLRightInline{Prado R, West M. Time series: Modeling, computation, and
inference. Boca Raton: CRC Press; 2010. }

\leavevmode\hypertarget{ref-hyndman2008forecasting}{}%
\CSLLeftMargin{30. }
\CSLRightInline{Hyndman R, Koehler AB, Ord JK, Snyder RD. Forecasting
with exponential smoothing: The state space approach. Springer Science
\& Business Media; 2008. }

\leavevmode\hypertarget{ref-makridakis2020m4}{}%
\CSLLeftMargin{31. }
\CSLRightInline{Makridakis S, Spiliotis E, Assimakopoulos V. The {M4
Competition}: 100,000 time series and 61 forecasting methods.
International Journal of Forecasting. 2020;36: 54--74. }

\leavevmode\hypertarget{ref-gelman2020bayesian}{}%
\CSLLeftMargin{32. }
\CSLRightInline{Gelman A, Vehtari A, Simpson D, Margossian CC, Carpenter
B, Yao Y, et al. Bayesian workflow {[}Internet{]}. 2020. Available:
\url{http://arxiv.org/abs/2011.01808}}

\end{CSLReferences}

\nolinenumbers


\end{document}
