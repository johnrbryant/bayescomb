% Template for PLoS
% Version 3.5 March 2018
%
% % % % % % % % % % % % % % % % % % % % % %
%
% -- IMPORTANT NOTE
%
% This template contains comments intended
% to minimize problems and delays during our production
% process. Please follow the template instructions
% whenever possible.
%
% % % % % % % % % % % % % % % % % % % % % % %
%
% Once your paper is accepted for publication,
% PLEASE REMOVE ALL TRACKED CHANGES in this file
% and leave only the final text of your manuscript.
% PLOS recommends the use of latexdiff to track changes during review, as this will help to maintain a clean tex file.
% Visit https://www.ctan.org/pkg/latexdiff?lang=en for info or contact us at latex@plos.org.
%
%
% There are no restrictions on package use within the LaTeX files except that
% no packages listed in the template may be deleted.
%
% Please do not include colors or graphics in the text.
%
% The manuscript LaTeX source should be contained within a single file (do not use \input, \externaldocument, or similar commands).
%
% % % % % % % % % % % % % % % % % % % % % % %
%
% -- FIGURES AND TABLES
%
% Please include tables/figure captions directly after the paragraph where they are first cited in the text.
%
% DO NOT INCLUDE GRAPHICS IN YOUR MANUSCRIPT
% - Figures should be uploaded separately from your manuscript file.
% - Figures generated using LaTeX should be extracted and removed from the PDF before submission.
% - Figures containing multiple panels/subfigures must be combined into one image file before submission.
% For figure citations, please use "Fig" instead of "Figure".
% See http://journals.plos.org/plosone/s/figures for PLOS figure guidelines.
%
% Tables should be cell-based and may not contain:
% - spacing/line breaks within cells to alter layout or alignment
% - do not nest tabular environments (no tabular environments within tabular environments)
% - no graphics or colored text (cell background color/shading OK)
% See http://journals.plos.org/plosone/s/tables for table guidelines.
%
% For tables that exceed the width of the text column, use the adjustwidth environment as illustrated in the example table in text below.
%
% % % % % % % % % % % % % % % % % % % % % % % %
%
% -- EQUATIONS, MATH SYMBOLS, SUBSCRIPTS, AND SUPERSCRIPTS
%
% IMPORTANT
% Below are a few tips to help format your equations and other special characters according to our specifications. For more tips to help reduce the possibility of formatting errors during conversion, please see our LaTeX guidelines at http://journals.plos.org/plosone/s/latex
%
% For inline equations, please be sure to include all portions of an equation in the math environment.
%
% Do not include text that is not math in the math environment.
%
% Please add line breaks to long display equations when possible in order to fit size of the column.
%
% For inline equations, please do not include punctuation (commas, etc) within the math environment unless this is part of the equation.
%
% When adding superscript or subscripts outside of brackets/braces, please group using {}.
%
% Do not use \cal for caligraphic font.  Instead, use \mathcal{}
%
% % % % % % % % % % % % % % % % % % % % % % % %
%
% Please contact latex@plos.org with any questions.
%
% % % % % % % % % % % % % % % % % % % % % % % %

\documentclass[10pt,letterpaper]{article}
\usepackage[top=0.85in,left=2.75in,footskip=0.75in]{geometry}

% amsmath and amssymb packages, useful for mathematical formulas and symbols
\usepackage{amsmath,amssymb}

% Use adjustwidth environment to exceed column width (see example table in text)
\usepackage{changepage}

% Use Unicode characters when possible
\usepackage[utf8x]{inputenc}

% textcomp package and marvosym package for additional characters
\usepackage{textcomp,marvosym}

% cite package, to clean up citations in the main text. Do not remove.
% \usepackage{cite}

% Use nameref to cite supporting information files (see Supporting Information section for more info)
\usepackage{nameref,hyperref}

% line numbers
\usepackage[right]{lineno}

% ligatures disabled
\usepackage{microtype}
\DisableLigatures[f]{encoding = *, family = * }

% color can be used to apply background shading to table cells only
\usepackage[table]{xcolor}

% array package and thick rules for tables
\usepackage{array}

% create "+" rule type for thick vertical lines
\newcolumntype{+}{!{\vrule width 2pt}}

% create \thickcline for thick horizontal lines of variable length
\newlength\savedwidth
\newcommand\thickcline[1]{%
  \noalign{\global\savedwidth\arrayrulewidth\global\arrayrulewidth 2pt}%
  \cline{#1}%
  \noalign{\vskip\arrayrulewidth}%
  \noalign{\global\arrayrulewidth\savedwidth}%
}

% \thickhline command for thick horizontal lines that span the table
\newcommand\thickhline{\noalign{\global\savedwidth\arrayrulewidth\global\arrayrulewidth 2pt}%
\hline
\noalign{\global\arrayrulewidth\savedwidth}}


% Remove comment for double spacing
%\usepackage{setspace}
%\doublespacing

% Text layout
\raggedright
\setlength{\parindent}{0.5cm}
\textwidth 5.25in
\textheight 8.75in

% Bold the 'Figure #' in the caption and separate it from the title/caption with a period
% Captions will be left justified
\usepackage[aboveskip=1pt,labelfont=bf,labelsep=period,justification=raggedright,singlelinecheck=off]{caption}
\renewcommand{\figurename}{Fig}

% Use the PLoS provided BiBTeX style
% \bibliographystyle{plos2015}

% Remove brackets from numbering in List of References
\makeatletter
\renewcommand{\@biblabel}[1]{\quad#1.}
\makeatother



% Header and Footer with logo
\usepackage{lastpage,fancyhdr,graphicx}
\usepackage{epstopdf}
%\pagestyle{myheadings}
\pagestyle{fancy}
\fancyhf{}
%\setlength{\headheight}{27.023pt}
%\lhead{\includegraphics[width=2.0in]{PLOS-submission.eps}}
\rfoot{\thepage/\pageref{LastPage}}
\renewcommand{\headrulewidth}{0pt}
\renewcommand{\footrule}{\hrule height 2pt \vspace{2mm}}
\fancyheadoffset[L]{2.25in}
\fancyfootoffset[L]{2.25in}
\lfoot{\today}

%% Include all macros below

\newcommand{\lorem}{{\bf LOREM}}
\newcommand{\ipsum}{{\bf IPSUM}}


% Pandoc citation processing
\newlength{\csllabelwidth}
\setlength{\csllabelwidth}{3em}
\newlength{\cslhangindent}
\setlength{\cslhangindent}{1.5em}
% for Pandoc 2.8 to 2.10.1
\newenvironment{cslreferences}%
  {}%
  {\par}
% For Pandoc 2.11+
\newenvironment{CSLReferences}[2] % #1 hanging-ident, #2 entry spacing
 {% don't indent paragraphs
  \setlength{\parindent}{0pt}
  % turn on hanging indent if param 1 is 1
  \ifodd #1 \everypar{\setlength{\hangindent}{\cslhangindent}}\ignorespaces\fi
  % set entry spacing
  \ifnum #2 > 0
  \setlength{\parskip}{#2\baselineskip}
  \fi
 }%
 {}
\usepackage{calc} % for calculating minipage widths
\newcommand{\CSLBlock}[1]{#1\hfill\break}
\newcommand{\CSLLeftMargin}[1]{\parbox[t]{\csllabelwidth}{#1}}
\newcommand{\CSLRightInline}[1]{\parbox[t]{\linewidth - \csllabelwidth}{#1}\break}
\newcommand{\CSLIndent}[1]{\hspace{\cslhangindent}#1}

\usepackage{setspace}\doublespacing
\usepackage{subcaption}



\usepackage{forarray}
\usepackage{xstring}
\newcommand{\getIndex}[2]{
  \ForEach{,}{\IfEq{#1}{\thislevelitem}{\number\thislevelcount\ExitForEach}{}}{#2}
}

\setcounter{secnumdepth}{0}

\newcommand{\getAff}[1]{
  \getIndex{#1}{}
}

\providecommand{\tightlist}{%
  \setlength{\itemsep}{0pt}\setlength{\parskip}{0pt}}

\begin{document}
\vspace*{0.2in}

% Title must be 250 characters or less.
\begin{flushleft}
{\Large
\textbf\newline{Supporting information for `A Bayesian approach to
combining multiple information sources: Estimating and forecasting
childhood obesity in
Thailand'} % Please use "sentence case" for title and headings (capitalize only the first word in a title (or heading), the first word in a subtitle (or subheading), and any proper nouns).
}
\newline
% Insert author names, affiliations and corresponding author email (do not include titles, positions, or degrees).
\\
\textsuperscript{}\\
\bigskip
\bigskip
\end{flushleft}
% Please keep the abstract below 300 words

% Please keep the Author Summary between 150 and 200 words
% Use first person. PLOS ONE authors please skip this step.
% Author Summary not valid for PLOS ONE submissions.

\linenumbers

% Use "Eq" instead of "Equation" for equation citations.
\hypertarget{additional-direct-estimates-of-obesity-prevalence}{%
\section{Additional direct estimates of obesity
prevalence}\label{additional-direct-estimates-of-obesity-prevalence}}

\begin{figure}[h]
\includegraphics{out/figS1}
\caption{\textbf{Direct estimates of obesity prevalence by age, region, and urban-rural residence for males, based on schools data}}
\end{figure}

\hypertarget{converting-nhes-data-into-effective-counts}{%
\section{Converting NHES data into effective
counts}\label{converting-nhes-data-into-effective-counts}}

Here we describe how we convert the raw National Health Examination
Survey data into effective counts.

Let \(o_{i}\) be 1 if the \(i\)th respondent in an NHES sample is obese,
and 0 otherwise; let \(w_{i}\) be the survey weight for that respondent;
let \(\eta_{as}\) denote the set of all respondents in age group \(a\)
and sex \(s\); and let \(\tilde{n}_{as}\) be the number of respondents.
We calculate \[
  p_{as} = \frac{\sum_{i \in \eta_{as}} w_i o_i}{\sum_{i \in \eta_{as}} w_i},
\] the obesity prevalence within that combination of age, sex, and year,
estimated using survey weights. We also calculate the variance in
weights \[
  V_{as} = \frac{\sum_{i \in \eta_{as}}(w_i - \bar{w}_{as})^2}{\tilde{n}_{as}-1},
\] and define design effect \[
d_{as} = 1 + \frac{V_{as}}{\bar{w}_{as}^2}.
\] The value \(d_{as}\) measures the extent to which sampling in age-sex
group \(as\) departs from simple random sampling. The higher the value
of \(d_{as}\), the greater the departure. Let \(d^*\) denote the median
value for \(d_{as}\). We define effective sample size as
\(n_{as} = \tilde{n}_{as}/d^*\) and define the effective number of obese
respondents as \(y_{as} = p_{as} n_{as}\). Before applying our model, we
round \(n_{as}\) and \(y_{as}\) to obtain whole numbers.

This procedure is identical to that in {[}1{]}, except that we use the
median design effect, rather than the mean, on the grounds that the
median is more robust to outliers.

\hypertarget{dropping-the-age-specific-bias-term-from-the-data-model-for-schools}{%
\section{Dropping the age-specific bias term from the data model for
schools}\label{dropping-the-age-specific-bias-term-from-the-data-model-for-schools}}

Fig 2 below shows national estimates and forecasts when the data model
for schools does not have an age-specific bias term. Fig 3b in the main
text shows the equivalent estimates and forecasts when the data model
for schools does have an age-specific bias term. Without the bias term,
there is one-off change in apparent prevalences for children aged 2--4
when moving from the period covered by the NHES and HDTC data to the
period covered by the schools data. The widths of the credible intervals
are also reduced. The one-off change appears implausible. Given that
there is uncertainty about age-specific biases, the wider credible
intervals of the main model are, we believe, more appropriate.

\begin{figure}[h]
\includegraphics{out/figS2}
\caption{\textbf{Estimates and forecast of obesity prevalence using a data model for schools that does not include an age-specific bias term}}
\end{figure}

\hypertarget{converting-who-obesity-estimates-into-effective-counts}{%
\section{Converting WHO obesity estimates into effective
counts}\label{converting-who-obesity-estimates-into-effective-counts}}

Our procedure for converting WHO obesity estimates into effective counts
is a modified version of the procedure for turning NHES data into
effective counts.

Let \(p_{cst}\) be the estimated prevalence for sex \(s\) in country
\(c\) in year \(t\). Let \(w_{cst}\) be the width of the associated 95\%
confidence interval. The width of a 95\% confidence interval should
equal approximately 4 standard deviations. If we treat \(p_{cst}\) as
being derived from the effective number of obese respondents divided by
the effective number of respondents \(n'_{cst}\), then one standard
deviation equals \(\sqrt{n'_{cst} p_{cst} (1 - p_{cwt})}\). We solve the
equation \begin{equation}
  w_{cst} = 4 \sqrt{n'_{cst} p_{cst} (1 - p_{cwt})}
\end{equation} to obtain a value for \(n_{cst}\). Individual values of
\(n'_{cst}\) can be very noisy, so we set \(n_{cwt}^{\text{W}}\) equal
to the median of \(n'_{cst}\) across \(s\) and \(t\). We then set
\(y_{cst}^{\text{W}} = p_{cst} n_{cwt}^{\text{W}}\).

\hypertarget{deriving-prior-distributions-from-who-data}{%
\section{Deriving prior distributions from WHO
data}\label{deriving-prior-distributions-from-who-data}}

Having fitted our model to the WHO data we obtain a sample
\(\tau_{\beta}^{\text{WHO}(k)}, k = 1, \cdots, K\) from the posterior
distribution for \(\tau_{\beta}^{\text{WHO}}\). We calculate
\(S^2 = \sum_{k=1}^K (\tau_{\beta}^{\text{WHO}(k)})^2\), and use
\(N(0, S^2)\) as our initial prior for \(\tau_{\beta}\) in (4) in the
main model. We obtain initial priors for \(\tau_{\alpha}\) and
\(\tau_{\delta}\) in the same way.

The fitted model for the WHO data also yields sample
\(\phi^{\text{WHO}(k)}\) from the posterior distribution for
\(\phi^{\text{WHO}}\). We set
\(\tilde{\phi}^{\text{WHO}} = \frac{\phi^{\text{WHO}(k) - 0.8}}{1 - 0.8}\),
and calculate \(m = \sum_{k = 1}^K \tilde{\phi}^{\text{WHO}(k)}\) and
\(V = \sum_{k = 1}^K (\tilde{\phi}^{\text{WHO}(k)} - m)^2 / K\). We set
\(a = m \left(\frac{m(1-m)}{v} - 1\right)\) and \(b = (1-m) a\), and use
\(\text{Beta}(a, b)\) as our initial prior for \(\phi\) in the main
model.

As discussed in the main text, we suspect that our initial posterior
distributions for \(\tau_{\beta}\), \(\tau_{\alpha}\),
\(\tau_{\delta}\), and \(\phi\) may understate true year-to-year
variability. We therefore construct modified versions of the priors that
allow for greater variability. Let \(k\) be a multiplier, which in
practice we set to 2 and 4. Our modified version of the prior for
\(\tau_{\beta}\) is simply \(N(0, (kS)^2)\), and similarly for
\(\tau_{\alpha}\) and \(\tau_{\delta}\). To obtain the modified version
of the prior for \(\phi\), we multiply \(V\) by \(k\).

\hypertarget{additional-results-from-models-4-and-5}{%
\section{Additional results from Models 4 and
5}\label{additional-results-from-models-4-and-5}}

\begin{figure}[h]
  \centering
  \begin{subfigure}[b]{\textwidth}
        \includegraphics{out/figS3a}
  \end{subfigure}
  \begin{subfigure}[b]{\textwidth}
        \includegraphics{out/figS3b}
  \end{subfigure}
  \caption{\textbf{Estimates and forecasts of obesity prevalence for males in urban areas, from Model 4 (top) and Model 5 (bottom)}}
\end{figure}

\begin{figure}[h]
  \centering
  \begin{subfigure}[b]{\textwidth}
        \includegraphics{out/figS4a}
  \end{subfigure}
  \begin{subfigure}[b]{\textwidth}
        \includegraphics{out/figS4b}
  \end{subfigure}
  \caption{\textbf{Estimates and forecasts of obesity prevalence for females in rural areas, from Model 4 (top) and Model 5 (bottom)}}
\end{figure}

\begin{figure}[h]
  \centering
  \begin{subfigure}[b]{\textwidth}
        \includegraphics{out/figS5a}
  \end{subfigure}
  \begin{subfigure}[b]{\textwidth}
        \includegraphics{out/figS5b}
  \end{subfigure}
  \caption{\textbf{Estimates and forecasts of obesity prevalence for males in rural areas, from Model 4 (top) and Model 5 (bottom)}}
\end{figure}

\clearpage

\hypertarget{references}{%
\section*{References}\label{references}}
\addcontentsline{toc}{section}{References}

\hypertarget{refs}{}
\begin{CSLReferences}{0}{0}
\leavevmode\hypertarget{ref-ghitza2013deep}{}%
\CSLLeftMargin{1. }
\CSLRightInline{Ghitza Y, Gelman A. Deep interactions with {MRP}:
Election turnout and voting patterns among small electoral subgroups.
American Journal of Political Science. 2013;57: 762--776. }

\end{CSLReferences}

\nolinenumbers


\end{document}
